\section{结论}
本研究主要以南纬35°以南的南大洋为研究区域。首先,我们发现包括MODIS在内的多个传统被动遥感数据,特别是叶绿素数据,在冬季高纬度地区存在缺失值的现象,这给$p\mathrm{CO_2}$的准确反演带来了挑战。然而,主动卫星成像的特点可以弥补被动遥感的缺点。基于Berenfeld等人\cite{bbp_Annual_2017,chen2017estimating}的研究表明,来自CALIPSO主动卫星遥感的$b_{bp}$数据与浮游植物生物量之间存在关系,因此我们用$b_{bp}$数据代替叶绿素数据来表征生物过程,以此建立反演模型。我们通过输入包括$b_{bp}$,SST,SSS,MLD,wind和$x\mathrm{CO_2}$在内的六种变量,建立了一个能准确反演南大洋$p\mathrm{CO_2}$的XGBoost模型。我们使用了预留的50条航次实测数据做独立测试,并对输入变量进行敏感性分析,来证明该模型的鲁棒性以及泛化能力。

该XGB模型可以用于长期观察南大洋的$p\mathrm{CO_2}$水平。它成功地实现了从2008年1月到2016年12月的南大洋$p\mathrm{CO_2}$的时空分布图像的建立,以及对海-气$\mathrm{CO_2}$通量的计算和预测。通过比较多种机器学习产品,我们发现了存在的差异。最后我们得到了以下结论:

(1)南大洋海表$p\mathrm{CO_2}$的时间与空间尺度的变化 

从时间变化来看,南大洋海表$p\mathrm{CO_2}$有着显著的季节变化,一般来说,在夏季较低,而在冬季达到最大值;研究区间内,平均海表$p\mathrm{CO_2}$的年际增长率为1.88μatm yr$^{-1}$,略低于大气的2.14μatmyr$^{-1}$,说明南大洋吸碳能力在逐年增加,从空间分布来看,南大洋$p\mathrm{CO_2}$低值主要集中在35°S到50°S的低纬度区域,意味着这里是吸收碳的主要区域。

(2)南大洋地区碳吸收能力的变化趋势

通过计算二氧化碳通量,我们能够量化分析南大洋地区的碳吸收能力变化,我们的结果表明南大洋在研究区间内能够吸收±1.37±0.04PgC yr$^{-1}$,而低纬度的南大洋地区能够吸收-0.96±0.02PgC yr$^{-1}$,占据总吸收量的70\%,由冰区和亚极地生物群落组成的高纬度区域约吸收了0.35±0.03PgCyr$^{-1}$;且南大洋每年吸收量约以0.016PgC的趋势在增长。

(3)多数回归反演产品低估了南大洋的吸碳量

多年来,众多学者基于回归方法来反演海表$p\mathrm{CO_2}$,且取得了一些进展,他们在南大洋冬季或是使用低值或气候态叶绿素替代,或是忽略叶绿素的作用,这些做法对结果都产生了一些影响,我们选择了几种代表性的产品进行了对比,发现各种产品在冬季高纬度区域的差异较大,多数产品存在着低估生活过程的吸收作用,从而导致了高估$p\mathrm{CO_2}$,进而认为冬季南大洋区域的放碳现象。

\section{本研究的主要创新点}
(1)融合主动遥感数据研究
该研究是通过集成主动遥感数据来进行的。这种研究方法的巨大价值在于,它通过将来自主动遥感的数据融合为模型中生物作用的学习变量,可以有效地解决极地地区被动遥感数据缺失的问题。相比于当前存在的且拥有连续时空分布的$p\mathrm{CO_2}$产品,这种方法能够提供更为准确和详细的信息。主动遥感卫星在冬季高纬度地区提供的是实测的遥感数据,这为我们进一步理解南大洋冬季碳汇变化趋势提供了极其重要的帮助。

(2)空间分辨率进一步提升

相较于其他的产品,我们的研究有着更为显著空间分辨优势。我们根据bbp数据进行了精细的分辨率调整,使得我们最后得到的$p\mathrm{CO_2}$产品数据分辨率得到了进一步的提升。这使得我们能够更为清晰地看到南大洋的时空变化趋势,为我们的进一步研究提供了宝贵的数据支持。

\section{本研究的不足与展望}
在本研究中,我们利用主动遥感数据和XGBoost算法建立了一个精度高、准确性强的南大洋海表$p\mathrm{CO_2}$反演模型。尽管该模型在算法对比和验证中显示出比以往的海表$p\mathrm{CO_2}$研究有显著改善,提供了南大洋地区碳吸收能力增加的定量证据,但仍存在一些不足和进一步研究的空间:

(1)主动遥感数据的利用有限

本文试图将主动遥感数据应用于极地区域,但由于主动遥感本身的成像过程,其数量和分布仍具有局限性,例如成像数量、测量观测范围等。此外,由于CALIOPSO卫星的绕地时间较短,其能够测量到白天和夜晚的数据,而本研究未考虑到昼夜差异,以及海洋昼夜碳汇量的差异,这可能是未来研究的方向。同时,CALIOPSO卫星自2006年发射以来,时间跨度较小,随着运行时间的增加和测量数据的积累,其数据的应用将会越来越广泛。

(2)遥感数据精度可以进一步提升

本研究的数据匹配是以月平均数据进行的,反演计算得到的$p\mathrm{CO_2}$时空分辨率为每月0.25°。虽然这在一定程度上保证了数据的充足,但考虑到时间跨度较大,遥感数据和实测数据之间可能存在误差。未来可以通过更精确的遥感产品进行建模,以显著提高模型的精度。

(3)实测数据有限

虽然本研究从遥感实测数据角度填补了冬季高纬度的数据缺失,但缺乏该区域的实测数据进行验证,可能需要更多的方法或新的测量手段来进行补充。





