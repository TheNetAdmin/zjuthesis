\cleardoublepage
% \chapternonum{致谢}
\chapternonumcenter{致谢}

岁月匆匆,三年的硕士生涯犹如白驹过隙,转瞬间即将画上句点。回首这段充满挑战与收获的旅程,心中不禁涌起万千感慨。这三年里,我沐浴在知识的海洋中,汲取着智慧的甘霖,积累了宝贵的经验。这段时光,如同一颗璀璨的明珠,闪耀在我人生的长河中。它不仅丰富了我的内心世界,更为我未来的征程奠定了坚实的基础。在即将踏上新的征程之际,我想借此机会向所有支持、鼓励和帮助过我的人致以最诚挚的感谢。

这篇论文的诞生,我首先要感谢我的导师乐成峰教授。从研究方向选取到模型建立,从数据分析到结论推敲,乐老师一路上给予了很大的帮助。乐老师不仅在学术上有高超的造诣和深入独特的知识理解,是我科研路上的良师,也在生活中言传身教,为我树立了极好的典范。在学术上他严谨治学、精益求精,对待每一个研究课题都充满了热情和执着。他总是鼓励我独立思考,勇于探索未知领域,让我在科研的道路上不断进步。而在生活中,他同样是我学习的榜样。他热爱生活、热爱体育运动,不仅自己积极参与,还常常鼓励我锻炼身体。乐老师在我困难时期伸出的援手,以及用自身成长经历给予的指引,都成为我人生中宝贵的财富。师恩如山,铭记于心。在此,我衷心祝愿乐老师在学术研究的道路上不断攀登新的高峰,取得更加辉煌的成就。您的教诲与恩情,我将永远铭记在心,并以此为动力,不断前行,追求更高的境界。

我要感谢我的朋友,那些热情而有趣的同门们,他们如同明亮的星光,在我乏味的科研生活中闪烁着耀眼的光芒。我们相互支持,相互鼓励,共同走过了科研探索的曲折与艰辛。无论是深夜的实验室灯光下还是课后的饭局上,那些与你们的讨论与笑声,仿佛成了我生活中最美妙的乐章。幽默风趣的何烁师兄,他的笑声常常在疲惫的研究时刻给予我勇气与力量,成为了我们团队的定心针;专注认真的张珂师姐,她的热情与坚持激励着我不断向前;博学多识的吴铭师兄,与他的交流中,我收获了许多新的思想火花和见解,如同拨云见日;见识渊博的邓培芳师姐,为我带来了无尽的启发与指导;自律严谨的古婷雨师姐,她自律的态度让我深深折服;乐观活泼的陆诗铭师姐,给整个团队带来了欢乐与活力;成熟稳重的蔡孙宾师兄,给予了我专业的健身指导,是运动场上的好伙伴;低调谦逊的屠泽斌师兄,总能提出自己独到的见解;初次相见在车站的牛星飞师兄,在科研上给予了我巨大的帮助;古灵精怪的胡雨焓师姐,让办公室充满了生机与趣味;与我同级但冷静性格沉稳的王鸿洋,则是我科研学习路上的好伙伴;思辨敏锐的张邦涵师弟和笑颜灿烂的何柯欣师妹,则是科研进度飞快;人美心善的林嘉晶师妹,为我们带来了许多快乐与正能量。正是因为你们的存在,我的研究生生活才被涂抹上了五彩斑斓的色彩,成为了一幅令人陶醉的画卷。

我要深深感谢浙江大学,这所我生活了七年的学府,见证了我的成长与蜕变。在这里,我领略到了浙大严谨的治学风气,使我养成了严谨求实的科研态度。同时,浙大丰富的学生活动也为我的大学生活增添了浓墨重彩的一笔,让我实现了全方面的成长。我还要特别感谢CC98平台。这个平台如同一个宝藏库,蕴藏着无尽的知识与智慧。在这里,我不仅收获了丰富的生活经验,宝贵的课程分享,更能够表达自己的思想感悟,与志同道合的朋友们交流碰撞。在浙大遇到的所有人,我都想说一句感谢,浦学聪、周嘉航是研究生期间为数不多仍保持联系的本科同学,我们时常交流自己的科研进度与生活琐事;还有我的室友顾慈航,不仅是我生活中的伙伴,还常常畅聊理想,分享彼此的梦想和追求。
浙大和CC98平台给予我的,不仅仅是知识和技能的积累,更重要的是一种精神层面的提升和成长。在这里,我学会了如何独立思考、如何面对挑战。

我要深深地感谢我的家人,是他们无私的爱与坚定的支持,让我能够走到今天,不断前行。父母用他们的双手,辛勤耕耘,将我送出大山,让我有机会看到这个世界的广阔与多彩。他们的理解与包容,是我成长道路上最坚实的后盾。家人的支持是我前行的动力,他们的期望是我不断进取的源泉。

最后的最后,我要感谢我自己。在这漫长的岁月里,我深知自己身上的缺点与不足,但我从未放弃过自我提升与努力。我会继续保持勇气、坚韧与毅力,迎接更多的挑战与机遇,书写更加精彩的人生篇章。



