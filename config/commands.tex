% common commands

\ifthenelse{\equal{\TwoSide}{true}}
{
    % TwoSide settings
    % Use default `\cleardoublepage'
}
{
    % OneSide settings
    \renewcommand{\cleardoublepage}{\clearpage}
}

\newcommand{\bful}[1]{{\bfseries\uline{#1}}}

% Commands to input body pages
\ifthenelse{\equal{\Degree}{undergraduate}}
{
    % undergraduate
    \newcommand{\inputpage}[1]{\input{./page/undergraduate/#1}}
    \newcommand{\inputbody}[1]{\input{./body/undergraduate/#1}}

    % TODO: how to switch to \section* without breaking current formats?
    \newcommand{\chapternonum}[1]
    {
        \cleardoublepage
        \phantomsection
        \addcontentsline{toc}{chapter}{#1}
        \begin{center}
            \bfseries \zihao{3} #1	
        \end{center}	
        \setcounter{section}{0}
    }

    \newcommand{\sectionnonum}[2][openright]
    {

        \ifthenelse{\equal{\TwoSide}{true}}{
            \ifthenelse{\equal{#1}{openright}}
                {\cleardoublepage}
                {\ifthenelse{\equal{#1}{openany}}{\clearpage}{}}
        }{
            \ifthenelse{\equal{#1}{openright}}
                {\clearpage}
                {\ifthenelse{\equal{#1}{openany}}{\clearpage}{}}
        }
        \phantomsection
        \addcontentsline{toc}{section}{#2}
        \begin{center}
            \bfseries \zihao{3} #2	
        \end{center}	
        \setcounter{subsection}{0}
    }
}
{
    % graduate
    \newcommand{\checkandinput}[2]
    {
        % Directory structure:
        % graduate/undergraduate            [Degree     ] [required        ]
        % |-- master/doctor                 [GradLevel  ] [optional        ]
        %     |-- general/cs/math/...       [MajorFormat] [optional        ]
        %         |-- cover.tex/toc.tex/... [TeX Files  ] [template defined]
        %
        % Search from bottom-level to top-level:
        % E.g:
        %     With such dir:
        %     graduate
        %     |-- cover.tex
        %     |-- toc.tex
        %     |-- abstract.tex
        %     |-- master
        %         |-- toc.tex
        %         |-- math
        %             |-- abstract.tex
        %     The input files are:
        %         graduate/cover.tex
        %         graduate/master/toc.tex
        %         graduate/master/math/abstract.tex
        \IfFileExists{./#1/graduate/\GradLevel/\MajorFormat/#2}
        {
            \input{./#1/graduate/\GradLevel/\MajorFormat/#2}
        }
        {
            \IfFileExists{./#1/graduate/\GradLevel/#2}
            {
                \input{./#1/graduate/\GradLevel/#2}
            }
            {
                \input{./#1/graduate/#2}
            }
        }
    }

    \newcommand{\inputpage}[1]
    {
        \checkandinput{page}{#1}
    }
    
    
    \newcommand{\inputbody}[1]
    {
        \checkandinput{body}{#1}
    }

    \newcommand{\chapternonum}[1]
    {
        \phantomsection
        \addcontentsline{toc}{chapter}{#1}
        \markboth{#1}{#1}
        \chapter*{#1}
    }

    \newcommand{\sectionnonum}[1]
    {
        \phantomsection
        \addcontentsline{toc}{section}{#1}
        \section*{#1}
    }
}

\ifthenelse{\equal{\Degree}{undergraduate}}
{
\newcommand{\signature}[1]
{
    \begin{flushright}
        \bfseries \zihao{-4}
        #1 \underline{\multido{}{5}{\quad}} \\
        \quad 年 \quad 月 \quad 日
    \end{flushright}
}

\newcommand{\eechecklistsig}[1]
{
  % For undergrad ee thesis
    \begin{center}
        \songti
        \zihao{-4}
        #1 \underline{\multido{}{7}{\quad}}
        \multido{}{3}{\quad} 
        检查日期:\underline{\multido{}{7}{\quad}}
    \end{center}
}

\DeclareDocumentCommand{\finaleval}{O{~} O{~} O{~} O{~} O{~}}
{
    \begin{table}[H]
        \centering \bfseries
        \begin{tabularx}{\textwidth}{|>{\fangsong}c
                                     |>{\fangsong}X<{\centering}
                                     |>{\fangsong}X<{\centering}
                                     |>{\fangsong}X<{\centering}
                                     |>{\fangsong}X<{\centering}
                                     |>{\fangsong}c|}
            \hline
            \makecell{成绩\\比例}
            & \makecell{\ifthenelse{\equal{\Type}{thesis}}{文献综述}{中期报告} \\(10\%)}
            & \makecell{开题报告\\(15\%)}
            & \makecell{外文翻译\\(5\%)}
            & \ifthenelse{\equal{\Type}{thesis}}{毕业论文质量及答辩(70\%)}{毕业设计质量及答辩(70\%)}
            & \makecell{总评\\成绩} \\

            \hline
            \multirow{2}*{分值}
            & \multirow{2}*{\zihao{4}#1}
            & \multirow{2}*{\zihao{4}#2}
            & \multirow{2}*{\zihao{4}#3}
            & \multirow{2}*{\zihao{4}#4}
            & \multirow{2}*{\zihao{4}#5} \\

            ~ & ~ & ~ & ~ & ~ & ~ \\
            \hline
        \end{tabularx}
    \end{table}
}

\DeclareDocumentCommand{\eefinaleval}{O{~} O{~} O{~} O{~} O{~} O{~}}
{
  \begin{center}
    \begin{tabular}{| >{\songti \zihao{5}}c
                    | >{\songti \zihao{5}}c
                    | >{\songti \zihao{5}}c
                    | >{\songti \zihao{5}}c 
                    | >{\songti \zihao{5}}c
                    | >{\songti \zihao{5}}c
                    | >{\songti \zihao{5}}c|}
    \hline
    \multirow{2}*{\textbf{成绩}}
    & \textbf{文献综述}
    & \textbf{开题报告}
    & \textbf{外文翻译}
    & \textbf{毕业设计(论文)质}
    & \textbf{成果评分}
    & \multirow{2}*{\textbf{总评成绩}} \\

    \textbf{比例}
    & \textbf{占(10\%)}
    & \textbf{占(15\%)}
    & \textbf{占(5\%)}
    & \textbf{量及答辩占(62\%)}
    & \textbf{占(8\%)}
    & ~ \\

    \hline

    \multirow{2}*{分值}
    & \multirow{2}*{\zihao{4}#1}
    & \multirow{2}*{\zihao{4}#2}
    & \multirow{2}*{\zihao{4}#3}
    & \multirow{2}*{\zihao{4}#4}
    & \multirow{2}*{\zihao{4}#5}
    & \multirow{2}*{\zihao{4}#6} \\

    ~
    & ~
    & ~
    & ~ 
    & ~ 
    & ~ 
    & ~ \\
    \hline
    \end{tabular}
  \end{center}
}

% `design` commands
\DeclareDocumentCommand{\designproposaleval}{O{~} O{~}}
{
    \begin{flushright}
        \begin{tabular}{| >{\fangsong \zihao{4}}c
                        | >{\fangsong \zihao{5}}c
                        | >{\fangsong \zihao{5}}c |}
            \hline
            \multirow{2}*{成绩比例}
            & 开题报告
            & 外文翻译 \\

            ~
            & 占(15\%)
            & 占(5\%) \\

            \hline

            \multirow{2}*{分值}
            & \multirow{2}*{\zihao{4}#1}
            & \multirow{2}*{\zihao{4}#2} \\
            
            ~
            & ~
            & ~ \\
            \hline
        \end{tabular}
    \end{flushright}
}

\DeclareDocumentCommand{\designmidcheckeval}{O{~}}
{
    \begin{flushright}
        \begin{tabular}{| >{\fangsong \zihao{4}}c
                        | >{\fangsong \zihao{5}}c |}
            \hline
            \multirow{2}*{成绩比例}
            & 中期报告 \\

            ~
            & (10\%) \\

            \hline

            \multirow{2}*{分值}
            & \multirow{2}*{\zihao{4}#1} \\

            ~
            & ~ \\
            \hline
        \end{tabular}
    \end{flushright}
}

% `thesis` commands
\ifthenelse{\equal{\MajorFormat}{ee}}
{
  \DeclareDocumentCommand{\thesisproposaleval}{O{~} O{~} O{~} O{~}}
  {
    \begin{center}
        \begin{tabular}{| >{\songti \zihao{4}}c
                        | >{\songti \zihao{5}}c
                        | >{\songti \zihao{5}}c
                        | >{\songti \zihao{5}}c 
                        | >{\songti \zihao{5}}c|}
            \hline
            \multirow{2}*{\ 成绩比例\ }
            & \ \ 文献综述\ \ 
            & \ \ 开题报告\ \ 
            & \ \ 外文翻译\ \ 
            & \multirow{2}*{\ \ \ \ \ 合\ \ \ \ 计\ \ \ \ \ } \\

            ~
            & 占(10\%)
            & 占(15\%)
            & 占(5\%)
            & ~ \\

            \hline

            \multirow{2}*{\ \ 分\ \ \ 值\ \ }
            & \multirow{2}*{\zihao{4}#1}
            & \multirow{2}*{\zihao{4}#2}
            & \multirow{2}*{\zihao{4}#3}
            & \multirow{2}*{\zihao{4}#4} \\

            ~
            & ~
            & ~
            & ~ 
            & ~ \\
            \hline
        \end{tabular}
    \end{center}
  }

  % For undergrad ee thesis
  \DeclareDocumentCommand{\eethesischecklistinfo}{O{~} O{~}}
  {
    \begin{center}
      \begin{tabular}{|>{\songti \zihao{5}}l
                      |>{\songti \zihao{5}}l
                        >{\songti \zihao{5}}l
                        >{\songti \zihao{5}}p{2.1cm}
                      |>{\songti \zihao{5}}l
                      |>{\songti \zihao{5}}p{2cm}|}
      \hline
      学生姓名                      & \multicolumn{1}{p{2.1cm}|}{\songti \zihao{5}\StudentName}       & \multicolumn{1}{l|}{学号} & \StudentID & 专业年级小班               &   \Class   \\ \hline
      指导教师                      & \multicolumn{1}{l|}{\songti \zihao{5}\AdvisorName}              & \multicolumn{1}{l|}{职称} & #1         & 学生联系电话               &   \Phone   \\ \hline
      \multirow{2}{*}{毕业设计题目}  & \multicolumn{3}{l|}{\multirow{2}{*}{\songti \zihao{5}\Title}}                                           & \multirow{2}{*}{评定成绩} & \multirow{2}{*}{#2} \\
                                   & \multicolumn{3}{l|}{}                                                                                   &                          &                   \\ \hline
      \end{tabular}
    \end{center}
  }

  \DeclareDocumentCommand{\eethesischecklist}{O{~} O{~}}
  {
    \begin{center}
      \begin{tabular}{|p{2cm}>{\songti \zihao{5}}p{8cm}|p{2cm}>{\songti \zihao{5}}p{1.4cm}|}
        \hline
        \multicolumn{2}{|c|}{\heiti\bfseries\zihao{-4}检\ 查\ 内\ 容}                                                          & \multicolumn{2}{l|}{\heiti\bfseries\zihao{-4}评价栏(打√)} \\ \hline
        \multicolumn{1}{|p{1.5cm}|}{}                                                      & 1.封面(使用统一封面,建议蓝色)     & \multicolumn{1}{>{\songti \zihao{5}}p{1.4cm}|}{较好}    & 一般 \\ \cline{2-4} 
        \multicolumn{1}{|l|}{}                                                             & 2.毕业设计(论文)承诺书            & \multicolumn{1}{>{\songti \zihao{5}}l|}{}              &     \\ \cline{2-4} 
        \multicolumn{1}{|c|}{\songti\bfseries\zihao{4}毕}                                  & 3.致谢                           & \multicolumn{1}{>{\songti \zihao{5}}l|}{}              &      \\ \cline{2-4} 
        \multicolumn{1}{|c|}{\songti\bfseries\zihao{4}业}                                  & 4.中文摘要、英文摘要(不超过 500 字) & \multicolumn{1}{>{\songti \zihao{5}}l|}{}              &     \\ \cline{2-4} 
        \multicolumn{1}{|c|}{\songti\bfseries\zihao{4}设}                                  & 5.目录(每项内容要对应标注页码)      & \multicolumn{1}{>{\songti \zihao{5}}l|}{}              &     \\ \cline{2-4} 
        \multicolumn{1}{|c|}{\songti\bfseries\zihao{4}计}                                  & 6.正文(从正文开始标注页码)         & \multicolumn{1}{>{\songti \zihao{5}}l|}{}              &     \\ \cline{2-4} 
        \multicolumn{1}{|c|}{\rotatebox[origin=c]{-90}{\songti\bfseries\zihao{4}(}}       & 7.参考文献                        & \multicolumn{1}{>{\songti \zihao{5}}l|}{}              &     \\ \cline{2-4} 
        \multicolumn{1}{|c|}{\songti\bfseries\zihao{4}论}                                  & 8.附录(可根据需要)                & \multicolumn{1}{>{\songti \zihao{5}}l|}{}              &     \\ \cline{2-4} 
        \multicolumn{1}{|c|}{\songti\bfseries\zihao{4}文}                                  & 9.作者简介                        & \multicolumn{1}{>{\songti \zihao{5}}l|}{}              &     \\ \cline{2-4} 
        \multicolumn{1}{|c|}{\rotatebox[origin=c]{-90}{\songti\bfseries\zihao{4})}}       & 10.毕业设计(论文)任务书           & \multicolumn{1}{>{\songti \zihao{5}}l|}{}              &     \\ \cline{2-4} 
        \multicolumn{1}{|l|}{}                                                             & 11.毕业设计(论文) 考核表          & \multicolumn{1}{>{\songti \zihao{5}}l|}{}              &     \\ \cline{2-4} 
        \multicolumn{1}{|l|}{}                                                             & 12.毕业设计(论文)专家评阅意见表     & \multicolumn{1}{>{\songti \zihao{5}}l|}{}              &     \\ \cline{2-4} 
        \multicolumn{1}{|l|}{}                                                             & 13.毕业设计(论文)现场答辩记录表     & \multicolumn{1}{>{\songti \zihao{5}}l|}{}              &     \\ \hline
        \end{tabular}
    \end{center}
  }
}
{
  \DeclareDocumentCommand{\thesisproposaleval}{O{~} O{~} O{~}}
  {
    \begin{flushright}
        \begin{tabular}{| >{\fangsong \zihao{4}}c
                        | >{\fangsong \zihao{5}}c
                        | >{\fangsong \zihao{5}}c
                        | >{\fangsong \zihao{5}}c |}
            \hline
            \multirow{2}*{成绩比例}
            & 文献综述
            & 开题报告
            & 外文翻译 \\

            ~
            & 占(10\%)
            & 占(15\%)
            & 占(5\%) \\

            \hline

            \multirow{2}*{分值}
            & \multirow{2}*{\zihao{4}#1}
            & \multirow{2}*{\zihao{4}#2}
            & \multirow{2}*{\zihao{4}#3} \\

            ~
            & ~
            & ~
            & ~ \\
            \hline
        \end{tabular}
    \end{flushright}
  }
}
}
{}

% From: https://tex.stackexchange.com/questions/395856/switching-tocdepth-in-the-middle-of-a-document
\newcommand{\changelocaltocdepth}[1]{%
  \addtocontents{toc}{\protect\setcounter{tocdepth}{#1}}%
  \setcounter{tocdepth}{#1}%
}